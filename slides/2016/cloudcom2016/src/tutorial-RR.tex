% =============================================================================
% File:  tutorial-RR.tex --
% Author(s): Sebastien Varrette (Sebastien.Varrette@uni.lu)
% Time-stamp: <Mon 2016-12-12 08:51 svarrette>
%
% Copyright (c) 2016 Sebastien Varrette<Sebastien.Varrette@uni.lu>
%
% For more information:
% - LaTeX: http://www.latex-project.org/
% - Beamer: https://bitbucket.org/rivanvx/beamer/
% - LaTeX symbol list:
% http://www.ctan.org/tex-archive/info/symbols/comprehensive/symbols-a4.pdf
% =============================================================================

\documentclass[t]{beamer}
% \documentclass[draft]{beamer}

\usepackage{_style}
% \usepackage{rotating}
\usepackage{figlatex}
\usepackage{smartdiagram}
% Biblio
\usepackage[style=verbose,autocite=footnote,maxnames=2,babel=hyphen,hyperref=true,abbreviate=false,mcite,backend=biber]{biblatex}
\setbeamerfont{footnote}{size=\tiny}

\addbibresource{biblio.bib}

% \renewcommand{\footnotesize}{\tiny}
% Removes icon in bibliography
\setbeamertemplate{bibliography item}{}


% The key part to use my theme -- if you precise nothing, the image that
% illustrate the slides is assumed to be images/slides_image.jpg
\usetheme[image=images/logo_ULHPC.pdf]{Falkor}

% Not integrated in my theme as not everybody wants that
\AtBeginSection[]
{
  \frame{
    \frametitle{Summary}
    {\scriptsize\tableofcontents[currentsection]}
  }
}
\AtBeginSubsection[]
{
  \frame{
    \frametitle{Summary}
    %\begin{multicols}{2}
      {\scriptsize\tableofcontents[currentsection,currentsubsection]}
    %\end{multicols}
  }
}

\graphicspath{{images/}{images/author_reader_rr/}} % Add this directory to the searched paths for graphics


%%%%%%%%%% Header %%%%%%%%%%%%
\title{Reproducible Research at the Cloud Era}
\subtitle{Overview, Hands-on and Open Challenges}

\author[Sebastien Varrette]{S\'ebastien Varrette, PhD}
\institute[University of Luxembourg]{\vspace*{-1em}
  Parallel Computing and Optimization Group (\href{http://pcog.uni.lu}{PCOG}),
  University of Luxembourg (\href{http://www.uni.lu}{UL}), Luxembourg\\[1em]

  \begin{tcolorbox}\centering
    \url{http://RR-tutorials.rtfd.io}
  \end{tcolorbox}

  \textbf{\alert{Before the tutorial starts}:} Visit\\
  \url{https://goo.gl/l9mCsM}\\
  for \textit{preliminary setup instructions}!
}

% Mandatory to **declare** a logo to be placed on the bottom right -- normally the
% university logo. ADAPT ACCORDINGLY:
\pgfdeclareimage[height=0.8cm]{logo}{images/logo_UL.pdf}

\date{}

%%%%%%%%%%%%% Body %%%%%%%%%%%%%%%
\begin{document}

\begin{frame}
  \vspace{2.5em}
  \titlepage
\end{frame}

% ......
\frame{
  \frametitle{Summary}
  {\scriptsize
    \tableofcontents
  }
}

% ___________________
\input{_me.md}
\input{_preamble.md}
\input{_pre-install.md}


% ____________________________________________________________
\section{Introduction and Motivating Examples}
\input{_intro.md}

%____________________________________________________________
\section{Reproducible Research}
\input{_reproducible_research.md}

%...................................................
\subsection{Easy-to \{read|take|share\} Docs}
\input{_docs.md}

%...................................................
\subsection{Sharing Code and Data}
\input{_sharing.md}
\input{_git.md}

%...................................................
\subsection{Mastering your [reproducible] environment}
\input{_environment.md}
\input{_vagrant.md}
\input{_software.md}
\input{_docker.md}


% % % %
% % Dissemination Platforms
% %
% % Workflow Tracking and Research Environments:
% % - VisTrails
% % - Galaxy
% % - Sumatra
% % - CDE
% % - Synapse...
% %
% % Embedded Publishing
% % - Figshare, ActivePapers, Elsevier executable paper
% % - Verifiable Computational Research
% % - knitR
% % - Sweave
% %
% % Good Practice for Setting up a Laboratory Notebook
% % - Step 0: Taking Notes
% % \input{__docs.md}
% % - Step 1: Sharing Code and Data
% % Git, Reprozip
% % - Step 2: Literate programming
% % with LaTeX and R: Sweave, knitR...
% % Ipython/Jupyter Notebook
% %
% % Automatic testing
% % - Bats, Travis, gitlab-CI, codeclimate....

%____________________________________________________________
\section{Conclusion}
\input{_conclusion.md}
%\input{_not_cover}
% Open Challenges
\input{_challenges.md}


% ======================== END =========================
\section*{Thank you for your attention...}
\frame{
\frametitle{Questions?}
  % ~~~~~~~~~~~~~~
\begin{columns}
  \column{0.5\textwidth}
    %   \emph{Contact}\\
  {\tiny
  \emph{Sebastien Varrette}\\
  ~~~~ \textit{mail:} \href{mailto:Sebastien.Varrette@uni.lu}{Sebastien.Varrette@uni.lu}\\
  ~~~~ Office E-007\\
  ~~~~ Campus Kirchberg\\
  ~~~~ 6, rue Coudenhove-Kalergi\\
  ~~~~ L-1359 Luxembourg

}
  \column{0.5\textwidth}
    %   \scalebox{8}{\emph{?}}
  \includegraphics[width=1.5in]{question.jpg}
\end{columns}
  % Below is the table of content over 2 columns
\vfill
\begin{multicols}{2}
  {\tiny \tableofcontents}
\end{multicols}

}

\newcounter{finalframe}
\setcounter{finalframe}{\value{framenumber}}

% % .......
% \frame{
% \frametitle{}
% \vfill
% \centering \LARGE Appendix\footnote{notice the slide number below...}
% \vfill
% }

\setcounter{framenumber}{\value{finalframe}}

\end{document}

% ~~~~~~~~~~~~~~~~~~~~~~~~~~~~~~~~~~~~~~~~~~~~~~~~~~~~~~~~~~~~~~~~
% eof
%
% Local Variables:
% mode: latex
% mode: flyspell
% mode: visual-line
% TeX-master: "tutorial-RR.tex"
% End:
